\documentclass{article}
\usepackage{booktabs}
\usepackage{amsmath}
\usepackage{float}
\usepackage{caption} % This must match your .bib file name exactly
\usepackage[english]{babel}
\usepackage{natbib}  % or another citation package

% Set page size and margins
% Replace `letterpaper' with`a4paper' for UK/EU standard size
\usepackage[letterpaper,top=2cm,bottom=2cm,left=3cm,right=3cm,marginparwidth=1.75cm]{geometry}
\usepackage{threeparttable}

% Useful packages
\usepackage{amsmath}
\usepackage{graphicx}

\usepackage[colorlinks=true, allcolors=blue]{hyperref}

\title{Coronavirus: Impact on Stock Prices and Growth Expectations}
\author{Amy Wang, Charlotte Zhou}

\begin{document}
\maketitle
\begin{abstract}
  In this project, we aim to replicate and validate the findings from the study Coronavirus: Impact on Stock Prices and Growth Expectations \cite{gormsen2020coronavirus}.
   Our primary goal is to reproduce the results presented in Figure 1, Figure 5, and Table 1. However, we discovered that Figure 5 and Table 1 rely on OTC market data. 
   According to the original paper Equity Yields by \cite{van2013equity}, data prior to 2008 was sponsored by Goldman Sachs and BNP Paribas. After we contacted Professor Ralph Koijen for access to this dataset but were denied,
    we shifted our approach to use data from 2008 onward.

  Because the S\&P dividend futures market data on Bloomberg only dates back to 2015, we have even 
  fewer data points for our analysis. Nonetheless, our findings show trends consistent with the original study, and we are satisfied with the progress made so far.
  \end{abstract}


\section{Introduction}
This project replicates the pivotal findings in the article. Our 
objective is to replicate Figure 1, Figure 5 and Table 1. Utilizing Bloomberg 
data for S\&P 500 dividend futures and index data, we not only include analysis from January 2020 to August 
2020 time frame, but also include data up to March 2025. This 
replication effort not only underscores the importance of empirical 
validation in financial research but also confronts the challenges 
of data availability and methodological adaptation inherent in such 
scholarly pursuits.
\section{Replicated Figure 1}

After we pulled market indices data for the covid period we are interested in, we did the following
data transformation to calculate prices from the 30 year yield. 
\[
\text{Price} 
= \frac{100}
  {\bigl(1 + \times \text{Yield}\bigr)^{30}}.
\]

See the replicated graph: 

\begin{figure}[H]
  \centering
  \includegraphics[width=0.8\textwidth]{../_output/figures/figure1.png} 
  \caption{Replicated Figure 1}
  \label{fig:example_image}
\end{figure}


\section{Methodology for Figure 5 and Table 1}
The regression we ran for Table 1 is as follows: 
\[
e_{it}^{(n)} = \frac{1}{n}ln(\frac{D_t}{F_t^{(n)}})
\]
where n is measured in years. So we later run a pooled regression of realized dividend growth rates on the S\&P 500, 
the Euro Stoxx 50, and Nikkei 225 onto the 2-year realized yield of the associated index. 
\[
\Delta_1 D_{i,t} 
= \beta_{0i}^D 
+ \beta_{1}^D \, e_{it}^{(2)} 
+ \epsilon_{i,t+4},
\]
where t is measured in quarters, and i refers to either S\&P 500, 
the Euro Stoxx 50, and Nikkei 225
\section{Replication for Table 1 and Figure 5}
See the replicated table 1:
\begin{table}[H]
  \centering
  \caption{Predictive Regressions of Dividend Growth on Dividend Yields}
  \input{../_output/tables/table1_results.tex}
  \label{tab:your_label}
  \end{table}

Given that we don't have that many data points (original paper has 143 data points), 
we've seen a much lower R-squared number. 
Later then we use those parameters to replicate figure 5, which shows the dynamics of expected dividend growht in the US, EU and Japan.
 
\begin{figure}[H]
  \centering
  \includegraphics[width=0.8\textwidth]{../_output/figures/paper_figure5_combined.png} 
  \caption{Replicated Figure 5}
  \label{fig:figure_5}
\end{figure}
Even though the parameter estimates were quite different, graph shows very similar trend to what was presented in the paper.
So we later extend this to the current market status.

\begin{figure}[H]
  \centering
  \includegraphics[width=0.8\textwidth]{../_output/figures/updated_figure5_combined.png}
  \caption{Replicated Figure 5 until now}
  \label{fig:figure_5_current}
\end{figure}

\section{Additional Summary Stats}
We also created summary statistics for the three source data's daily returns: S\&P 500 Index, Euro Stoxx 50, and Nikkei 225 so that
our reader can better understand those three markets on a broader level. 

\begin{table}[H]
  \centering
  \caption{Summary Statistics for Source Data}
  \input{../_output/tables/additional_stats.tex}
  \label{tab:table3}
  \end{table}


\raggedright
\section{Successes and Challenges}
In our efforts to replicate the original study, we closely followed its methodology and logic. Our automated data extraction process proved especially valuable, allowing us to retrieve data in a manner consistent with the original authors’ approach. Despite the absence of OTC market data, we were still able to produce Figure 5 and Table 1, demonstrating similar trends to those reported in the study.

\bibliographystyle{apalike}
\bibliography{references}

\end{document}
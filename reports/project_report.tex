\documentclass{article}
\usepackage{booktabs}
\usepackage{amsmath}
\usepackage{float}
\usepackage{caption} % This must match your .bib file name exactly
\usepackage[english]{babel}
\usepackage{natbib}  % or another citation package

% Set page size and margins
% Replace `letterpaper' with`a4paper' for UK/EU standard size
\usepackage[letterpaper,top=2cm,bottom=2cm,left=3cm,right=3cm,marginparwidth=1.75cm]{geometry}

% Useful packages
\usepackage{amsmath}
\usepackage{graphicx}

\usepackage[colorlinks=true, allcolors=blue]{hyperref}

\title{Coronavirus: Impact on Stock Prices and Growth Expectations}
\author{Amy Wang, Charlotte Zhou}

\begin{document}
\maketitle
\begin{abstract}
  In this project, we endeavor to replicate and validate the 
findings in the study ``Coronavirus: Impact on Stock Prices and Growth Expectations''
 The core objective is to reproduce the results presented in Figure 1,Figure 5, and Table 1. We soon realized that 
 Figure 5 and Table 1 are based on OTC market data. From the original paper ``Equity yields ``, the data before 08 was sponsored by
 Goldman Sachs and BNP Paribas. After requesting this dataset from Professor Ralph Koijen and received a no, we decided
 to replicate the result using data after 2008. Of course, S&P dividend future market only have data since 2015 (on Bloomberg), so
 we have fewer data points to work with.  But results show similar trends and we are happy with the progress we've made. 
 
  \end{abstract}


\section{Introduction}
This project replicates the pivotal findings in the article. Our 
objective is to replicate Table 1 and Table 2. Utilizing Bloomberg 
data for S\&P 500 futures and zero-coupon yields from the Federal 
Reserve, we not only include analysis from January 1988 to June 
2017 time frame, but also include data up to January 2024. This 
replication effort not only underscores the importance of empirical 
validation in financial research but also confronts the challenges 
of data availability and methodological adaptation inherent in such 
scholarly pursuits.

\raggedright
\section{Successes and Challenges}
In our endeavor to replicate the seminal study, we successfully adhered to the paper's logic and replication methodology. Our access
to and automation of data extraction were particularly noteworthy; we were able to seamlessly retrieve data, mirroring the process utilized
by the original authors. Moreover, our replication of $pd_t$ and $pr_t$
revealed only minimal discrepancies in comparison to the published results.
Significantly, we achieved complete automation of the replication and
table generation processes, all operationalized through a single 'doit'
command. The thoroughness of our replication was affirmed by the
successful passage of all unit tests within reasonable tolerance levels.\newline
\newline
Despite these successes, we encountered several obstacles. Most notably, 
we were unable to access the Fama-Bliss database, which was the 
source of the 1-year zero-coupon yields in the original research. To 
circumvent this, we sourced the data from the Federal Reserve's website, 
although this introduced some methodological differences. The yields 
from the Federal Reserve are spot rates, which may not align with the 
non-standard rates used in the Fama-Bliss database, potentially leading 
to discrepancies in our calculated $pr_t$ and the ultimate findings. 
\bibliographystyle{apalike}
\bibliography{references}

\end{document}